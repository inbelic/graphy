\documentclass[12pt, letterpaper]{article}

\usepackage{graphicx}

\title{Graph Theory with Applications: Notes}
\author{Christopher Finn Plummmer}
\date{December 5, 2022}

\begin{document}

\maketitle

\newpage

\section{Graphs and Subgraphs}

\subsection{Excercises}

The first definitions we encounter on this journey are those of a
\textit{simple, identical and isomorphic} graphs. A simple graph can be
described as a graph that does not have loops or if there exists at most one
edge between any distinct pair of vertices. An identical graph with respect
to another is a graph such that all vertices are labelled the same and all
edges between vertices are the same. In contrast, a graph is isomorphic to
another graph if there exists a bijection of vertices to vertices such that
we keep all corresponding edges. This raises our first programming task:
define an abstract graph data type. Then define three functions that will
denote if the graph is simple, two graphs are identical and if two graphs
are isomorphic.


\end{document}
